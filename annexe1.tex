\chapter{Annexe B: Radial exampl}
\paragraph*{}
This example shows:
\begin{itemize}[font=\color{black}, label=\maltese]
\item A radial graph
\item The selected node is "S", as "X" is a direct linked node (level 1), Y a level 2 linked node and F
must not be depicted in the picture because it is not linked to a level 1, F is a level 3 node
\item Relations with arrows, unidirectional "in" and unidirectional "out" and bidirectional
\end{itemize}
\begin{figure}[h]
\centering
\includegraphics[scale=0.5, frame]{RadialGraph.png}
\caption{Radial Graph}
\end{figure}
\newpage
\textbf{Radial}
The chosen node is the center of the graph\\
The linked nodes are organized on circles (aka orbit) as per distance level\\
Rules: when a node "A" is linked with two (or more) nodes of situated on different orbits, the node is
placed on the lowest orbit as possible. For example:
\begin{itemize}
\item Node$_$ a is the central node
\item Node$_$ a is linked with Node$_$ b and node$_$ c \rightarrow Node_b and Node_c are on orbit 1
\item If Node$_$b and Node$_$ c are linked it creates a transversal relation of the same level
\item Node$_$ b is linked with Node$_$ d \rightarrow node$_$ d is on orbit 2
\item Node$_ $c is linked with Node$_$ e \rightarrow node$_$ e is on orbit 2
\item Node$_$ e is linked with Node$_$ f \rightarrow node$_$ f is on orbit 3 at this point, when continuing the exploration we notice that:
\begin{itemize}
\item Node$_$ f is linked with Node$_$b \rightarrow node$_$ f is promoted to orbit 2 Please note that the
exploration does not take into account the link direction property, that can make the
exploration more recursive.
\end{itemize}
\end{itemize}
\paragraph*{}
Biradial and triradial might be an option to be developed in a later stage if useful. A biradial will consist in selecting two nodes and a triradial three nodes. These nodes might not have a common a common link. The selected nodes must be displayed on the screen harmoniously (in triangle for a triradial graph). This feature might be required to show how two or threee nodes are linked together