\chapter*{Conclusion}
\phantomsection
%\addcontentsline{toc}{chapter}{General Conclusion}
\addstarredchapter{General Conclusion}
 \pagestyle{fancy}
 \thispagestyle{\renewcommand{\headrulewidth}{0pt}}
\rhead{}
\lhead{}
\lettrine[lines=2,lhang=0.44,lraise=0,loversize=0.08,findent=-0.11em,slope=0.6em]%
{A}{} proverb is a simple and concrete saying popularly known and revered, which expresses a truth, based on common sense or the practical experience of humanity. The proverb "A picture is worth 1000 words" is one you have probably heard more than once.
\paragraph*{}
Informed decision making is the foundation upon which successful businesses are built. As a decision maker, one needs access to highly visual business intelligence tools that can help making the right decisions quickly. As organizations grow, so does the amount of collected information. If this data is delivered in spreadsheets or tabular reports, it becomes more and more challenging to find the patterns, trends and correlations necessary to well perform the required job.
\paragraph*{}
The practice of representing information visually is nothing new. Scientists, students, and analysts have been using data visualization for centuries to track everything from astrological phenomena to stock prices. Only recently, with the adoption of more sophisticated BI technology in the corporate world and the ever-increasing practice of data collection and data mining activities, has data visualization in the form of dashboards been used as an important presentation tool in business analysis. As a result, the use of dashboards in making quick and accurate business decisions has become an essential requirement for remaining competitive. The common forms of data visualization are Basic Charts and Status Indicators. This remains very useful however the nature of data managed by the organizations has evolved and became more and more complex. More advanced data visualizations forms were needed. This includes scatter graphs, bubble charts, spark line charts, geographical maps, tree maps, Pareto charts, and many others. These more sophisticated visualizations are designed to display data in ways tailored to a specific function or industry.
\paragraph*{}
Decision makers need to interact with their data to expose trends, highlight opportunities and raise red flags quickly and accurately. Their data should answer key questions and provide insight into issues that contribute directly to the decision making process. Presenting this data visually and adding contextual information to complement the analysis process not only makes it quicker and easier to pinpoint areas of opportunities and concern, but also enables decision makers to take action with their data. Successful data visualization provides the ability to expose problem areas and communicate those problems universally. Not being able to clearly identify and share your discoveries to back up your decisions can mean the difference between taking appropriate and decisive action and losing momentum or failing to act.
\paragraph*{}
Using data visualization to display large amounts of data is nothing new. However, its value and use in making business decisions is often overlooked or poorly implemented. The key to success in using data visualization is ensuring that: the best and most appropriate types of visualizations are used; the data is always put into perspective with contextual information allowing for the information to be universally understood; and that the data being measured within the data visualization enables the user to take action based on the observations being made. With a good set of visuals that keep these key success factors in mind, decisions can be made more quickly and with more confidence so that your business can continue to grow.
\paragraph*{}
It is in this exciting and intellectually challenging and motivating setting that I did my internship graduation. The past months have been very instructive for me. Stradefi SA has offered me opportunities to learn and develop myself in many areas. I gained a lot of experience, especially in the Data Visualization field as well as the project planning and monitoring. I also developed my skills in the design and programming of significant software solutions. A lot of the tasks and activities that I have worked on during my internship are familiar with what I?m studying at the moment. I worked in many areas where I did different work.
I also learned what working in a team really means. The sense of the term "Deadlines" has another dimension in my mind. I acquired a the meaning of the "rigorous work", things they would not teach me of in college: only confrontation with the reality of work allows us to understand and grow. 